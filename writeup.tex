\documentclass{article}

%\usepackage{amsthm}%
\usepackage{amsmath}
\usepackage{amssymb}
\usepackage{amsfonts}
\usepackage[margin=1in]{geometry}
\usepackage{enumerate}
\usepackage{verbatim}
\usepackage{graphicx}
\usepackage{hyperref}
\usepackage{url}
\usepackage[]{algorithm2e}
\usepackage[T1]{fontenc}
\newenvironment{proof}{\noindent{\bf Proof:} \hspace*{1mm}}{
        \hspace*{\fill} $\Box$ }
\newenvironment{proof_of}[1]{\noindent {\bf Proof of #1:}
        \hspace*{1mm}}{\hspace*{\fill} $\Box$ }
\newenvironment{proof_claim}{\begin{quotation} \noindent}{
        \hspace*{\fill} $\diamond$ \end{quotation}}

\newtheorem{thm}{Theorem}
\newtheorem{lemma}[thm]{Lemma}

\newcommand{\mbf}{\mathbf}
\newcommand{\mrm}{\mathrm}

\newcommand\abs[1]{\left|#1\right|}
\newcommand\given[1][]{\:#1\vert\:}
\newcommand{\legendre}[2]{\genfrac{(}{)}{}{}{#1}{#2}}
\setcounter{tocdepth}{1}
\newtheorem{theorem}{Theorem}
\pagenumbering{gobble}

\title{\textsc{EE16B} Final Project Writeup}
\author{%
        Christopher Branner-Augmon \\
          \texttt{ee16b-amb}
        \and
        Ahmed Husain \\
          \texttt{ee16b-amy}
        \and
        Peter Manohar \\
          \texttt{ee16b-aif}
        \and
        Pratyush Mishra \\
        \texttt{ee16b-aci}
        \and
        Amanda Tomlinson\\
        \texttt{ee16b-amu}
      }

\begin{document}
\maketitle
\section*{PCA Classification}

We tried to pick words that had distinct lengths and distinct vowel sounds.

\paragraph{Words used.}
We initially tried only varying the length of the words, but found that
consistently fitting longer words in the short recording interval was difficult.
So, to make make each word more distinct, we tried varying the length and sound
of vowels in our words. After some experimentation, we settled on the words
``port'', ``halibut'', ``google'', and ``starboard''.

We chose ``port'' because it was short, unlike all the other words.
``Starboard'', on the other hand, is the longest of our words. To vary the
vowels we used, we picked ``google'' and ``halibut'' to include a long ``o''
and short ``a'' sound.

We also said each word in a very unusual manner to emphasize differences, thus
ensuring that the sampled waveforms would be clearly distinct.  This helped
improve the consistency of our classification. For example, when saying the
word starboard, we added an inflection (specifically, a rising ``ou'' sound) at
the end of the word. Without this, our classification was not very accurate.

We obtained distinct clusters for ``halibut'', ``google'', and ``starboard'',
after projecting our data onto the principal components.  The data points
corresponding to ``port'' were spread out, but did not interfere with the other
clusters.

When we implemented the k-means classification on the launchpad, we modified
the algorithm slightly. Instead of using the same threshold distance for all
the clusters, we set different threshold distances for each cluster. We did
this because the cluster corresponding to ``port'' was significantly more spread out
than the other clusters. Without this change we were unable to classify
``port'' correctly, since the data point was rarely within the uniform minimum
distance.

\section*{What we learned}

The project gave us a better understand of PCA and k-means. We also learned how
to implement control systems, and struggled with debugging the car. Issues regarding
the car came from both sides of the system, both through the software and hardware side.

An important lesson that we learned when trying to classify different words 
was how a change in hardware can result in us having to change the software
as well. This happened when we had to switch the microphone circuits.

We also learned that somestimes the circuit schematic and values that are given
to us may not always work with our design. It goes to show that many 
non-idealities can prop up and force us to make minor changes in order
for the system to work properly.
Our demo is available at \url{https://www.youtube.com/watch?v=LVntcSF0hhY}.

\end{document}
