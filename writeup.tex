\documentclass{article}

%\usepackage{amsthm}%
\usepackage{amsmath}
\usepackage{amssymb}
\usepackage{amsfonts}
\usepackage[margin=1in]{geometry}
\usepackage{enumerate}
\usepackage{verbatim}
\usepackage{graphicx}
\usepackage{hyperref}
\usepackage[]{algorithm2e}
\usepackage[T1]{fontenc}
\newenvironment{proof}{\noindent{\bf Proof:} \hspace*{1mm}}{
	\hspace*{\fill} $\Box$ }
\newenvironment{proof_of}[1]{\noindent {\bf Proof of #1:}
	\hspace*{1mm}}{\hspace*{\fill} $\Box$ }
\newenvironment{proof_claim}{\begin{quotation} \noindent}{
	\hspace*{\fill} $\diamond$ \end{quotation}}

\newtheorem{thm}{Theorem}
\newtheorem{lemma}[thm]{Lemma}

\newcommand{\mbf}{\mathbf}
\newcommand{\mrm}{\mathrm}

\newcommand\abs[1]{\left|#1\right|}
\newcommand\given[1][]{\:#1\vert\:}
\newcommand{\legendre}[2]{\genfrac{(}{)}{}{}{#1}{#2}}
\setcounter{tocdepth}{1}
\newtheorem{theorem}{Theorem}
\pagenumbering{gobble}
\begin{document}
\section*{PCA Classification}

We tried to pick words that had distinct lengths and distinct vowel sounds. We
figured that having these features would allow the PCA to classify well. We
initially tried only varying the length of the words, but we found that it was
hard to consistently fit in longer words in the recording interval. This lead
to us trying to use vowel sounds as a way to choose words. We ended up settling
on the words ``port'', ``halibut'', ``google'', and ``starboard''. 

``Port'' was chosen because it was short, and ``starboard'' was chosen because
it was long. We picked ``google'' and ``halibut'' to include a long ``o'' and
short ``a'' sound. We also said each word in a very unusual way to ensure that
the waveforms sampled would be very distinct, and our classification would
therefore consistent. For example, when saying the word starboard, we added an
inflection at the end of the word. Without this, our classification was not
very accurate.  We used PCA to classify the data, and we got a distinct cluster
for ``halibut'', ``google'', and ``starboard''. The points corresponding to
``port'' were spread out, but did not interfere with the other clusters. 

When we
implemented the classification on the launchpad, we made a slight change in the
algorithm.  Instead of outputting the word corresponding to the closest cluster
within some uniform minimum distance, we set a minimum distance for each
cluster individually. We did this because the spread of port's cluster was
significantly higher than the other clusters. Without this change we were
unable to classify port because the signal was rarely within the uniform
minimum distance.

\section*{What we learned}

The project gave us a better understand of PCA and k-means. We also learned how
to implement control systems, and struggled with debugging the car.

\end{document}
